\chapter{Stochastic Calculus and Option Pricing}
\label{ch:stochastic_calculus}

\section{Introduction to Stochastic Calculus}
\label{sec:ito_intro}

Stochastic calculus extends classical calculus to functions of random processes. It is indispensable in modern financial mathematics, particularly for derivative pricing and risk management.

\subsection{Motivation: Why Classical Calculus Fails}
\label{subsec:motivation}

Consider a function $f(W_t)$ where $\{W_t\}$ is Brownian motion. Classical calculus suggests:
\[
df = f'(W_t) \, dW_t
\]

However, Brownian paths have unbounded variation, making the Riemann-Stieltjes integral undefined. The key insight is that $(dW_t)^2 = dt$ in a specific limiting sense, requiring a new calculus.

\section{The Itô Integral}
\label{sec:ito_integral}

\subsection{Construction of the Itô Integral}
\label{subsec:ito_construction}

\begin{definition}[Itô Integral]
For a stochastic process $\{H_t\}$ adapted to the filtration generated by Brownian motion $\{W_t\}$, the Itô integral is:
\[
I(t) = \int_0^t H_s \, dW_s = \lim_{n \to \infty} \sum_{i=0}^{n-1} H_{t_i}(W_{t_{i+1}} - W_{t_i})
\]
where the limit is in probability and the partition $0 = t_0 < t_1 < \cdots < t_n = t$ becomes arbitrarily fine.
\end{definition}

\textbf{Key Feature:} The integrand $H_s$ is evaluated at the \emph{left endpoint} $t_i$ (Itô convention), which ensures the integral is a martingale.

\subsection{Properties of the Itô Integral}
\label{subsec:ito_properties}

\begin{theorem}[Properties of Itô Integral]
Let $\{H_t\}$ and $\{K_t\}$ be adapted processes. Then:
\begin{enumerate}
    \item \textbf{Linearity:}
    \[
    \int_0^t (aH_s + bK_s) \, dW_s = a\int_0^t H_s \, dW_s + b\int_0^t K_s \, dW_s
    \]
    
    \item \textbf{Martingale Property:}
    \[
    \E\left[\int_0^t H_s \, dW_s\right] = 0
    \]
    
    \item \textbf{Itô Isometry:}
    \[
    \E\left[\left(\int_0^t H_s \, dW_s\right)^2\right] = \E\left[\int_0^t H_s^2 \, ds\right]
    \]
\end{enumerate}
\end{theorem}

The Itô isometry relates the variance of the stochastic integral to a deterministic integral, making it computable.

\subsection{Examples of Itô Integrals}
\label{subsec:ito_examples}

\begin{example}[Constant Integrand]
For constant $c$:
\[
\int_0^t c \, dW_s = c W_t \sim \N(0, c^2 t)
\]
\end{example}

\begin{example}[Brownian Motion as Integrand]
\[
\int_0^t W_s \, dW_s = \frac{1}{2}W_t^2 - \frac{1}{2}t
\]
Note the correction term $-\frac{t}{2}$, which does not appear in classical calculus!
\end{example}

\section{Itô's Lemma}
\label{sec:ito_lemma}

Itô's lemma is the fundamental theorem of stochastic calculus, analogous to the chain rule in ordinary calculus.

\subsection{One-Dimensional Itô's Lemma}
\label{subsec:ito_lemma_1d}

\begin{theorem}[Itô's Lemma—One Dimension]
Let $X_t$ satisfy the SDE:
\[
dX_t = \mu(t, X_t) \, dt + \sigma(t, X_t) \, dW_t
\]
and let $f(t, x)$ be a twice continuously differentiable function. Then $Y_t = f(t, X_t)$ satisfies:
\[
dY_t = \left(\frac{\partial f}{\partial t} + \mu \frac{\partial f}{\partial x} + \frac{1}{2}\sigma^2 \frac{\partial^2 f}{\partial x^2}\right) dt + \sigma \frac{\partial f}{\partial x} \, dW_t
\]
\end{theorem}

\textbf{Compact Notation:}
\[
df = f_t \, dt + f_x \, dX_t + \frac{1}{2}f_{xx}(dX_t)^2
\]
where $(dX_t)^2 = \sigma^2 \, dt$ (Itô correction).

\subsection{Derivation Sketch}
\label{subsec:ito_derivation}

Using Taylor expansion:
\[
df \approx f_t \, dt + f_x \, dX + \frac{1}{2}f_{xx}(dX)^2 + \text{higher order terms}
\]

Since $dX = \mu \, dt + \sigma \, dW$:
\begin{align}
(dX)^2 &= (\mu \, dt + \sigma \, dW)^2 \\
&= \mu^2 (dt)^2 + 2\mu\sigma \, dt \, dW + \sigma^2 (dW)^2 \\
&= \sigma^2 \, dt + o(dt)
\end{align}

where we use $(dt)^2 = o(dt)$, $dt \, dW = o(dt)$, and $(dW)^2 = dt$.

\subsection{Key Examples}
\label{subsec:ito_examples_detailed}

\begin{example}[Exponential Function]
Let $X_t = W_t$ and $f(x) = e^x$. Then:
\[
d(e^{W_t}) = e^{W_t} \, dW_t + \frac{1}{2}e^{W_t} \, dt
\]
Integrating:
\[
e^{W_t} = 1 + \int_0^t e^{W_s} \, dW_s + \frac{1}{2}\int_0^t e^{W_s} \, ds
\]
\end{example}

\begin{example}[Quadratic Function]
Let $f(x) = x^2$. For $dX_t = \mu \, dt + \sigma \, dW_t$:
\[
d(X_t^2) = 2X_t \, dX_t + \frac{1}{2} \cdot 2 \cdot \sigma^2 \, dt = 2X_t \, dX_t + \sigma^2 \, dt
\]
The term $\sigma^2 \, dt$ is the Itô correction absent in classical calculus.
\end{example}

\subsection{Multi-Dimensional Itô's Lemma}
\label{subsec:ito_multidim}

\begin{theorem}[Itô's Lemma—Multiple Dimensions]
Let $X_t = (X_t^1, \ldots, X_t^n)$ satisfy:
\[
dX_t^i = \mu^i(t, X_t) \, dt + \sum_{j=1}^m \sigma^{ij}(t, X_t) \, dW_t^j
\]
For $f(t, x^1, \ldots, x^n) \in C^{1,2}$:
\[
df = \frac{\partial f}{\partial t} \, dt + \sum_{i=1}^n \frac{\partial f}{\partial x^i} dX_t^i + \frac{1}{2}\sum_{i,j=1}^n \frac{\partial^2 f}{\partial x^i \partial x^j} \, dX_t^i dX_t^j
\]
where:
\[
dX_t^i dX_t^j = \sum_{k=1}^m \sigma^{ik}\sigma^{jk} \, dt
\]
\end{theorem}

\section{Stochastic Differential Equations}
\label{sec:sde}

\subsection{Definition and Existence}
\label{subsec:sde_definition}

\begin{definition}[Stochastic Differential Equation]
An SDE is an equation of the form:
\[
dX_t = \mu(t, X_t) \, dt + \sigma(t, X_t) \, dW_t, \quad X_0 = x_0
\]
or in integral form:
\[
X_t = x_0 + \int_0^t \mu(s, X_s) \, ds + \int_0^t \sigma(s, X_s) \, dW_s
\]
\end{definition}

\begin{theorem}[Existence and Uniqueness]
If $\mu$ and $\sigma$ satisfy:
\begin{enumerate}
    \item \textbf{Lipschitz condition:} There exists $K > 0$ such that for all $t, x, y$:
    \[
    |\mu(t,x) - \mu(t,y)| + |\sigma(t,x) - \sigma(t,y)| \leq K|x-y|
    \]
    \item \textbf{Linear growth:} There exists $C > 0$ such that:
    \[
    |\mu(t,x)| + |\sigma(t,x)| \leq C(1 + |x|)
    \]
\end{enumerate}
Then the SDE has a unique strong solution.
\end{theorem}

\subsection{Solving SDEs with Itô's Lemma}
\label{subsec:solving_sdes}

\begin{example}[Geometric Brownian Motion]
Solve:
\[
dS_t = \mu S_t \, dt + \sigma S_t \, dW_t, \quad S_0 = S_0
\]

\textbf{Solution:} Let $Y_t = \log S_t$. By Itô's lemma with $f(s) = \log s$:
\begin{align}
dY_t &= \frac{1}{S_t} dS_t - \frac{1}{2S_t^2}(dS_t)^2 \\
&= \mu \, dt + \sigma \, dW_t - \frac{1}{2}\sigma^2 \, dt \\
&= \left(\mu - \frac{\sigma^2}{2}\right) dt + \sigma \, dW_t
\end{align}

Integrating:
\[
Y_t = Y_0 + \left(\mu - \frac{\sigma^2}{2}\right)t + \sigma W_t
\]

Therefore:
\[
S_t = S_0 \exp\left(\left(\mu - \frac{\sigma^2}{2}\right)t + \sigma W_t\right)
\]
\end{example}

\section{The Black-Scholes-Merton Framework}
\label{sec:black_scholes}

\subsection{Model Assumptions}
\label{subsec:bsm_assumptions}

The Black-Scholes-Merton model makes the following assumptions:

\begin{enumerate}
    \item Stock price follows GBM: $dS_t = \mu S_t \, dt + \sigma S_t \, dW_t$
    \item Risk-free rate $r$ is constant
    \item No dividends
    \item No transaction costs or taxes
    \item Continuous trading possible
    \item Short selling allowed
    \item Markets are frictionless
\end{enumerate}

\subsection{Derivation of the Black-Scholes PDE}
\label{subsec:bsm_derivation}

Consider a European option with payoff $\Phi(S_T)$ at maturity $T$. Let $V(t, S_t)$ be the option value.

\textbf{Delta Hedging Portfolio:}
Construct a portfolio:
\[
\Pi_t = V(t, S_t) - \Delta_t S_t
\]
where $\Delta_t = \frac{\partial V}{\partial S}$ (the option's delta).

\textbf{Dynamics:} By Itô's lemma:
\[
dV = \left(\frac{\partial V}{\partial t} + \mu S \frac{\partial V}{\partial S} + \frac{1}{2}\sigma^2 S^2 \frac{\partial^2 V}{\partial S^2}\right) dt + \sigma S \frac{\partial V}{\partial S} \, dW_t
\]

The portfolio change is:
\[
d\Pi = dV - \Delta \, dS
\]

Substituting and choosing $\Delta = \frac{\partial V}{\partial S}$ eliminates the $dW_t$ term:
\[
d\Pi = \left(\frac{\partial V}{\partial t} + \frac{1}{2}\sigma^2 S^2 \frac{\partial^2 V}{\partial S^2}\right) dt
\]

\textbf{No-Arbitrage Condition:} A riskless portfolio must earn the risk-free rate:
\[
d\Pi = r\Pi \, dt = r\left(V - S\frac{\partial V}{\partial S}\right) dt
\]

Equating:
\[
\frac{\partial V}{\partial t} + \frac{1}{2}\sigma^2 S^2 \frac{\partial^2 V}{\partial S^2} = r\left(V - S\frac{\partial V}{\partial S}\right)
\]

\begin{theorem}[Black-Scholes PDE]
The option value $V(t,S)$ satisfies:
\[
\frac{\partial V}{\partial t} + rS\frac{\partial V}{\partial S} + \frac{1}{2}\sigma^2 S^2 \frac{\partial^2 V}{\partial S^2} - rV = 0
\]
with terminal condition $V(T, S) = \Phi(S)$.
\end{theorem}

\subsection{Black-Scholes Formula}
\label{subsec:bs_formula}

\begin{theorem}[Black-Scholes Formula for European Options]
The prices of European call and put options are:
\begin{align}
C(S, K, T, r, \sigma) &= S N(d_1) - K e^{-rT} N(d_2) \\
P(S, K, T, r, \sigma) &= K e^{-rT} N(-d_2) - S N(-d_1)
\end{align}
where:
\begin{align}
d_1 &= \frac{\ln(S/K) + (r + \sigma^2/2)T}{\sigma\sqrt{T}} \\
d_2 &= d_1 - \sigma\sqrt{T} = \frac{\ln(S/K) + (r - \sigma^2/2)T}{\sigma\sqrt{T}}
\end{align}
and $N(\cdot)$ is the standard normal CDF.
\end{theorem}

\subsection{Put-Call Parity}
\label{subsec:put_call_parity}

\begin{theorem}[Put-Call Parity]
For European options with the same strike $K$ and maturity $T$:
\[
C - P = S - Ke^{-rT}
\]
\end{theorem}

\begin{proof}
Consider two portfolios:
\begin{itemize}
    \item Portfolio A: Long call + $Ke^{-rT}$ in cash
    \item Portfolio B: Long put + one share of stock
\end{itemize}

At maturity $T$:
\begin{itemize}
    \item If $S_T > K$: Both worth $S_T$
    \item If $S_T \leq K$: Both worth $K$
\end{itemize}

By no-arbitrage, they must have the same value today.
\end{proof}

\section{The Greeks}
\label{sec:greeks}

The Greeks measure sensitivities of option prices to various parameters, essential for risk management and hedging.

\subsection{First-Order Greeks}
\label{subsec:first_order_greeks}

\subsubsection{Delta}

\begin{definition}[Delta]
\[
\Delta = \frac{\partial V}{\partial S}
\]
\end{definition}

For Black-Scholes:
\begin{align}
\Delta_{\text{call}} &= N(d_1) \in [0, 1] \\
\Delta_{\text{put}} &= N(d_1) - 1 \in [-1, 0]
\end{align}

\textbf{Interpretation:} A delta of 0.6 means a \$1 increase in stock price increases option value by approximately \$0.60.

\textbf{Hedging:} To delta-hedge, hold $-\Delta$ shares per option sold.

\subsubsection{Vega}

\begin{definition}[Vega]
\[
\mathcal{V} = \frac{\partial V}{\partial \sigma}
\]
\end{definition}

For Black-Scholes (same for call and put):
\[
\mathcal{V} = S N'(d_1) \sqrt{T} = S \phi(d_1) \sqrt{T}
\]
where $\phi(\cdot) = N'(\cdot)$ is the standard normal PDF.

\textbf{Interpretation:} Vega measures sensitivity to volatility. Options are always long vega (increase in volatility increases option value).

\subsubsection{Theta}

\begin{definition}[Theta]
\[
\Theta = \frac{\partial V}{\partial t} = -\frac{\partial V}{\partial \tau}, \quad \tau = T - t
\]
\end{definition}

For Black-Scholes call:
\[
\Theta_{\text{call}} = -\frac{S \phi(d_1) \sigma}{2\sqrt{T}} - rKe^{-rT}N(d_2)
\]

\textbf{Interpretation:} Time decay. Long options lose value as time passes (negative theta).

\subsubsection{Rho}

\begin{definition}[Rho]
\[
\rho = \frac{\partial V}{\partial r}
\]
\end{definition}

For Black-Scholes:
\begin{align}
\rho_{\text{call}} &= KT e^{-rT} N(d_2) > 0 \\
\rho_{\text{put}} &= -KT e^{-rT} N(-d_2) < 0
\end{align}

\textbf{Interpretation:} Calls benefit from higher interest rates, puts suffer.

\subsection{Second-Order Greeks}
\label{subsec:second_order_greeks}

\subsubsection{Gamma}

\begin{definition}[Gamma]
\[
\Gamma = \frac{\partial^2 V}{\partial S^2} = \frac{\partial \Delta}{\partial S}
\]
\end{definition}

For Black-Scholes (same for call and put):
\[
\Gamma = \frac{\phi(d_1)}{S \sigma \sqrt{T}} = \frac{N'(d_1)}{S \sigma \sqrt{T}}
\]

\textbf{Interpretation:} Gamma measures convexity. High gamma means delta changes rapidly, requiring frequent rehedging.

\textbf{Properties:}
\begin{itemize}
    \item $\Gamma > 0$ for long options (convexity benefit)
    \item $\Gamma$ is highest for ATM options near expiration
    \item Delta hedging has $\Gamma$ risk (P\&L from delta-hedged portfolio $\approx -\frac{1}{2}\Gamma(\Delta S)^2$)
\end{itemize}

\subsection{Greek Relationships}
\label{subsec:greek_relationships}

\begin{theorem}[Black-Scholes Greek Relationship]
For any derivative satisfying the Black-Scholes PDE:
\[
\Theta + rS\Delta + \frac{1}{2}\sigma^2 S^2 \Gamma = rV
\]
\end{theorem}

This relationship allows computing one Greek from the others.

\section{Implied Volatility}
\label{sec:implied_volatility}

\subsection{Definition}
\label{subsec:iv_definition}

\begin{definition}[Implied Volatility]
The implied volatility $\sigma_{\text{imp}}$ is the volatility parameter that, when input into the Black-Scholes formula, yields the observed market price:
\[
V_{\text{market}} = V_{\text{BS}}(S, K, T, r, \sigma_{\text{imp}})
\]
\end{definition}

\subsection{Numerical Computation}
\label{subsec:iv_computation}

Since there is no closed form for $\sigma_{\text{imp}}$, numerical methods are required:

\subsubsection{Newton-Raphson Method}

Iteratively solve:
\[
\sigma_{n+1} = \sigma_n - \frac{V_{\text{BS}}(\sigma_n) - V_{\text{market}}}{\mathcal{V}(\sigma_n)}
\]
where $\mathcal{V} = \frac{\partial V_{\text{BS}}}{\partial \sigma}$ is vega.

\textbf{Advantages:} Quadratic convergence when close to solution.

\subsubsection{Bisection Method}

Uses the monotonicity of $V_{\text{BS}}$ in $\sigma$:
\begin{enumerate}
    \item Initialize bounds: $\sigma_{\min}, \sigma_{\max}$
    \item Compute $\sigma_{\text{mid}} = (\sigma_{\min} + \sigma_{\max})/2$
    \item Update bounds based on $V_{\text{BS}}(\sigma_{\text{mid}}) \lessgtr V_{\text{market}}$
    \item Repeat until convergence
\end{enumerate}

\textbf{Advantages:} Guaranteed convergence, robust.

\subsection{Volatility Smile and Skew}
\label{subsec:vol_smile}

Empirically, implied volatility is not constant across strikes, violating Black-Scholes assumptions.

\begin{definition}[Volatility Smile]
The volatility smile is the graph of implied volatility $\sigma_{\text{imp}}(K)$ versus strike price $K$ (or moneyness $K/S$).
\end{definition}

\textbf{Typical Patterns:}
\begin{itemize}
    \item \textbf{Equity options:} Volatility skew (higher IV for low strikes)
    \item \textbf{FX options:} Symmetric smile (higher IV for OTM options)
    \item \textbf{Index options:} Pronounced skew (crash fear premium)
\end{itemize}

\textbf{Economic Explanations:}
\begin{enumerate}
    \item \textbf{Leverage effect:} Falling prices increase volatility (for equities)
    \item \textbf{Fat tails:} Market returns have heavier tails than log-normal
    \item \textbf{Jump risk:} Discontinuous price movements
    \item \textbf{Supply/demand:} Hedging demand for certain strikes
\end{enumerate}

\subsection{Volatility Surface}
\label{subsec:vol_surface}

\begin{definition}[Volatility Surface]
The volatility surface is the function:
\[
\sigma_{\text{imp}}(K, T)
\]
varying with both strike $K$ and time to maturity $T$.
\end{definition}

\textbf{Term Structure:} Volatility often varies with maturity:
\begin{itemize}
    \item Short-term: More sensitive to current market conditions
    \item Long-term: Reverts to long-run average
\end{itemize}

\section{Extensions and Advanced Topics}
\label{sec:extensions}

\subsection{American Options}
\label{subsec:american_options}

American options can be exercised at any time $t \in [0,T]$. The value satisfies a free boundary problem.

\textbf{Optimal Stopping:} The value is:
\[
V(t, S) = \sup_{\tau \in [t,T]} \E[e^{-r(\tau - t)}\Phi(S_\tau) | S_t = S]
\]

No closed-form solution exists (except special cases like American put on non-dividend stock).

\subsection{Path-Dependent Options}
\label{subsec:path_dependent}

\subsubsection{Asian Options}

Payoff depends on average price:
\[
\Phi = \max\left(\frac{1}{T}\int_0^T S_t \, dt - K, 0\right)
\]

Reduces sensitivity to manipulation near expiration.

\subsubsection{Barrier Options}

Knock-out option becomes worthless if $S_t$ hits barrier $B$:
\[
\Phi = \max(S_T - K, 0) \cdot \mathbb{1}_{\{\max_{0 \leq t \leq T} S_t < B\}}
\]

Cheaper than vanilla options due to knockout risk.

\subsection{Jump-Diffusion Models}
\label{subsec:jump_diffusion}

To capture sudden price movements, add jumps to GBM:
\[
dS_t = \mu S_t \, dt + \sigma S_t \, dW_t + S_{t-} dJ_t
\]
where $J_t$ is a jump process (e.g., compound Poisson).

\textbf{Merton Model:}
\[
\log(1 + J) \sim \N(\mu_J, \sigma_J^2), \quad N_t \sim \text{Poisson}(\lambda t)
\]

This completes our treatment of stochastic calculus and option pricing. We have developed the mathematical machinery for derivative valuation and risk management, which forms the theoretical foundation for quantitative finance. The next chapter applies these tools alongside machine learning techniques to model and forecast financial time series.
